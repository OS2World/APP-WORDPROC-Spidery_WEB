% Copyright 1989 by Norman Ramsey and Odyssey Research Associates
% To be used for research purposes only
% For more information, see file COPYRIGHT in the parent directory

% spiderman.tex, with apologies to Stan Lee 

\documentstyle[11pt]{article}
\setcounter{secnumdepth}{0}
\newcommand{\syntax}[1]{\mbox{$\langle\hbox{\sl #1\/}\rangle$}}
\newcommand{\produces}{\mbox{${}::={}$}}
\newcommand{\opt}[1]{$[$#1$]$}
\newcommand{\BS}{\relax}
\chardef\BS=`\\ % backslash in a string

\title{A {Spider} User's Guide}
\author{Norman Ramsey\\Department of Computer Science\\Princeton University}
\date{July 1989}
\newcommand {\WEB}{{\tt WEB}}

\begin{document}
\maketitle

\section{Introduction}
Donald Knuth developed the {\tt WEB} system of structured documentation 
as part of the {\TeX} project~\cite{knuth:literate-programming}.
{\WEB} enables a programmer to divide his or her program into chunks (called
{\em modules}), to associate text with each chunk, and to present the
chunks in in any order.
In Knuth's implementation, the chunks are pieces of PASCAL programs,
and the chunks are formatted using {\TeX}.

  The {\tt WEB} idea suggests a way of combining {\em any}
programming language with {\em any} document formatting language,
but until recently there was no software support for writing anything
but PASCAL programs using {\tt WEB}.
In~1987, Silvio Levy rewrote the {\tt WEB} system in C for C,
while retaining {\TeX} as the formatting language~\cite{levy:cweb}.
I have has modified Levy's implementation by removing the parts
that  make C the target programming language, and I have
 added a third tool, {Spider}, which complements {\tt WEAVE}
and {\tt TANGLE}.
{Spider} reads a description of a programming language, and writes
source code for a {\tt WEAVE} and {\tt TANGLE} which support that
language. 
Using {Spider}, a C~compiler, and an Awk~interpreter, an experienced
systems programmer can generate  a {\tt WEB} system for
an Algol-like language in a few hours.

This document explains how to use {Spider} to generate a {\WEB}
system for any programming language.
(The choice of programming language is limited only by the lexical
structure built into Spidery {\tt WEB}, as we shall see.)
You should consult the companion document, ``The Spidery {\WEB} system of
structured documentation,'' to learn how to use the generated {\WEB} system.

\paragraph{Prerequisites}
If you are going to use {Spider} to build a {\WEB} system,
you  should be comfortable using {\tt WEB}.
To get an idea how {\tt WEB} works, you should have read Knuth's
introductory article 
on {\WEB}~\cite{knuth:literate-programming}, as well as the {\WEB}
users' manual.
(The {\WEB} user's manual is pretty heavy going, so you may want to
consult the Bibliography for more introductory material on {\WEB}.
Wayne Sewell's {\it Weaving a Program: Literate Programming in {\tt
WEB}}
may be helpful~\cite{sewell:weaving}.)

In what follows we will assume that you know what {\tt WEAVE} and {\tt
TANGLE} are, what input they expect, and what output they produce.

\paragraph{Plan of this guide}
We'll begin with a review of weaving and tangling, so that we can get
an idea what is necessary to build a language-independent {\WEB}.
Then we'll present a discussion of the features of {Spider} that
tell {\WEB} about the programming language.
We'll define these in detail and give some examples, and then 
we'll close with a complete description of the {Spider} language
and tools.

\section{How {\tt WEAVE} and {\tt TANGLE} see the world}
Both {\tt WEAVE} and {\tt TANGLE} operate on the same input, a {\WEB}
file.
{\tt WEAVE} must examine this input and produce a {\TeX} text, while
{\tt TANGLE} must produce a program text from the same input.
The input consists of {\TeX} parts, definition parts, and code parts.
The {\TeX} parts are the easiest to consider: {\tt WEAVE} just copies
them and {\tt TANGLE} throws them away.
The definition parts are a bit more complicated: {\tt WEAVE}'s job is
to typeset them, while {\tt TANGLE} must remember the definitions and
expand them at the proper time.
The code parts are the most complex of all: {\tt WEAVE} must
prettyprint them, and {\tt TANGLE} must rearrange them into a coherent
program text.

\paragraph{Lexical analysis in {\WEB}}
Both {\tt WEAVE} and {\tt TANGLE} interpret the code parts as a stream
of {\em tokens}.
Since not all programming languages have the same tokens, it is 
{Spider}'s job to tell {\tt WEAVE} and {\tt TANGLE} how to tokenize the
input.%
\footnote{%
The current implementation of {\tt WEB}'s lexical analysis is limited.
It should be replaced with something using regular
expressions.% 
}
A Spidery {\WEB} system can recognize the following kinds of
tokens:
\begin{itemize}
\item identifiers
\item numeric and string constants
\item newlines
\item ``pseudo-semicolons'' (the token {\tt @;})
\item reserved words
\item non-alphanumeric tokens
\end{itemize}

{\tt TANGLE} rearranges these tokens into one long program
text, then writes out the program text token by token.
Normally, {\tt TANGLE} puts no white space between tokens, but it
will put blanks between adjacent identifier, reserved word, and
numeric constant tokens.
Thus the input
\begin{quote}
\tt if 0 > x-y then z := -1;
\end{quote}
will be written out as
\begin{quote}
\tt if 0>x-y then z:=-1;
\end{quote}
and not
\begin{quote}
\tt if0>x-ythenz:=-1;
\end{quote}
which wouldn't parse.
When it is desirable to have {\tt TANGLE} translate
the tokens differently, each token can be given a {\tt tangleto}
attribute, which specifies what  program text is printed out for that
token.
For example, 
the {\tt spider} file used to generate C~{\WEB} forces the {\tt =}
tokento be printed out as the string {\tt "=\ "}, because in C the string
{\tt "=-"} can be ambiguous.

{\tt WEAVE} must turn the token stream into a {\TeX} text that will
cause the code to be prettyprinted.
It does so in three steps:
\begin{enumerate}
\item 
{\tt WEAVE} turns each token into a {\em scrap}.
A scrap has two important properties: its syntactic 
{\em category} and its {\em translation}.
The categories are symbols in a prettyprinting grammar; that grammar
tells {\tt WEAVE} how to combine the scraps
with prettyprinting instructions.
The translations are the {\TeX} texts that will tell {\TeX} exactly
how to print the scraps.
\item
{\tt WEAVE} reduces the scrap stream by combining scraps according to
the productions of its prettyprinting grammar.
({\tt WEAVE} does a kind of shift-reduce parsing of program fragments.)
While combining the translations, {\tt WEAVE} adds {\TeX} text that
will cause indenting, outdenting, line breaking, and so on.
\item
Ideally, {\tt WEAVE} keeps reducing scraps until it has a single scrap with
a very long translation, but perhaps it will end up
with an irreducible sequence of scraps.
In any case, after no more reductions can be done, the translations of the
remaining scraps are output one at a time.
\end{enumerate}

\section{Using {Spider} to tell {\WEB} how to tokenize}
{Spider} divides tokens into two classes; reserved words and
other.
The reserved words as specified using the {\tt reserved} and {\tt ilk}
commands; the other tokens are specified using the {\tt token}
command.
(This somewhat unusual setup is dictated by the way {\tt
WEAVE} works; its advantage is that is is easy to define a whole group
of reserved words that will be treated identically.)
Here's how it works: the {\tt reserved} command designates
 a particular identifier as a reserved word, and says
what {\em ilk} it belongs to.
The {\tt token} and {\tt ilk} commands tell {\tt WEAVE} and {\tt
TANGLE} what to do with a particular token, or with all the reserved
words of a particular ilk.
For each token or ilk one can specify the {\em tangleto} field, the
token's {\em mathness} (whether it has to be typeset in math mode), and
its {\em category} and {\em translation} (for conversion to scraps).
All but the category can have defaults, set with the {\tt defaults}
command.
Choice of category names is up to the user.

We will discuss the tokenization commands more later when we present
the syntax of {Spider} in detail.
Meanwhile, 
here are some example tokenization commands from the {\tt spider} file
for~C:
\begin{verbatim}
token + category unorbinop
token - category unorbinop
token * category unorbinop
token = category equals translation <"\\leftarrow"> tangleto <"="-space>
token ~ category unop translation <"\\TI">
token & category unorbinop translation <"\\amp">
token ^ translation <"\\^"> category binop
token ? translation <"\\?"> category question
token % translation <"\\%"> category binop
token # translation <"\\#"> category sharp
token ! category unop translation <"\\neg">
token ( category lpar
token ) category rpar
token [ category lpar
token ] category rpar
token { translation <"\\{"> category lbrace
token } translation <"\\}"> category rbrace
token ++ category unop translation <"\\PP">
token -- category unop translation <"\\MM">
token != translation <"\\I"> category binop
token == translation <"\\S"> category binop
token && translation <"\\W"> category binop

ilk case_like category case
ilk int_like category int

reserved auto ilk int_like
reserved break ilk case_like
reserved case ilk case_like
reserved char ilk int_like
\end{verbatim}
These show the definitions of some of the tokens used in C.
Notice the {\tt tangleto} option is almost always left to default, and
the {\tt translation} option is often left to default.

Once the tokens are specified, and each has a {\tt tangleto} string,
we can almost construct a {\tt TANGLE} for the language. 
 Before we can construct a {\tt WEAVE}, we have to tell it how to
combine and reduce scraps.

\section{Using {Spider} to tell {\tt WEAVE} how to reduce scraps}
The most intricate part of {\tt WEAVE} is its mechanism for converting
programming language code into \TeX\ code.
{\tt WEAVE} uses a simple bottom-up parsing algorithm, since it
 must deal with fragmentary
constructions whose overall ``part of speech'' is not known.

The input is represented as a sequence of  {\em scraps}, 
where each scrap of information consists
of two parts, its {\em category} and its {\em translation}. 
The category
is essentially a syntactic class, and the translation represents
{\TeX} code.
Rules of syntax and semantics tell us how to
combine adjacent scraps into larger ones, and if we are lucky an entire
program text that starts out as hundreds of small scraps will join
together into one gigantic scrap whose translation is the desired \TeX\
code. 
If we are unlucky, we will be left with several scraps that don't
combine; their translations will simply be output, one by one.

The combination rules are given as context-sensitive productions that are
applied from left to right. Suppose that we are currently working on the
sequence of scraps $s_1\,s_2\ldots s_n$. We try first to find the longest
production that applies to an initial substring $s_1\,s_2\ldots\,$; but if
no such productions exist, we find to find the longest production
applicable to the next substring $s_2\,s_3\ldots\,$; and if that fails, we
try to match $s_3\,s_4\ldots\,$, et cetera.

A production applies if the category codes have a given pattern. For
example, if one of the productions is
$$\hbox{\tt open [ math semi <"\BS\BS,"-opt-5> ] -->
open math}$$ 
then it means that three consecutive scraps whose respective categories are
{\tt open}, {\tt math}, and {\tt semi} are con\-verted to two scraps whose categories
are {\tt open} and {\tt math}. 
 The {\tt open} scrap has not changed, while the string 
{\tt <"\BS\BS,"-opt-5>} 
indicates that the new {\tt math} scrap
has a translation composed of the translation of the original
{\tt math} scrap followed by the translation of the {\tt semi} scrap followed
by `{\tt \BS,}' followed by `{\tt opt}' followed by `{\tt5}'. (In the \TeX\ file,
this will specify an additional thin space after the semicolon, followed
by an optional line break with penalty 50.) 
Translations are enclosed in angle brackets, and may contain quoted
strings (using the C conventions to escape backslashes and so on), or
may contain special keywords.

Before giving examples of useful productions, we'll break to give the
detailed syntax of the {Spider} subset covered so far.



\section{Syntax of {\tt spider} files}

{Spider} is an Awk program which converts a description of a
language into C~code for {\tt WEAVE} and {\tt TANGLE}.
Since {Spider} is an Awk program, its input is a sequence of
lines, and all {Spider} commands must fit on one line.

\paragraph{Comments and blank lines}
Because {\em any} character sequence can be a token of a programming
language, we can't just designate a particular sequence as a ``begin
comment'' marker.
So in {Spider} there are no comments, only {\em comment lines}.
A comment line is one whose first character is ``{\tt \#}''.
The {Spider} processor ignores comment lines and blank lines.

\paragraph{Fields}
Each command in the {\tt spider} file consists of a sequence of {\em
fields}.
These are just the Awk fields, and they are separated by white space.
This feature of {Spider} (inherited from Awk) forbids the use of
white space within a field.

\subsection{Translations}
Most fields in a {Spider} file are simple identifiers, or perhaps
strings of non-alphanumeric characters.
The major exception is {\em translations}.
Translations are always surrounded by angle brackets ({\tt <>}),
and consist of a (possibly empty) list of translation pieces.
The pieces on a list are separated by dashes ({\tt -}).
A piece is one of:
\begin{itemize}
\item A quoted string.
This string may contain embedded quotes escaped by ``\verb+\+'', but
it {\em must not} contain embedded white space or an embedded dash.
\item The ``self'' marker, ``{\tt *}'',
 refers to the sequence of characters making up the token being
translated.
The self marker is permitted only in certain contexts, and its precise
meaning depends on the context.
\item A digit.
\item A key word.
The key words known to {Spider} are
\begin{description} 
\item [\tt space] Stands for one space ({\tt "\ "}).
\item[\tt dash] Stands for a dash ({\tt "-"}).
\end{description}
The other key words are passed on to {\tt WEAVE}.

{\tt WEAVE} recognizes the following key words:
\begin{description}
\item[\tt break\_space] denotes an optional line break or an en space;
\item[\tt force] denotes a line break;
\item[\tt big\_force] denotes a line break with additional vertical space;
\item[\tt opt] denotes an optional line break (with the continuation
line indented two ems with respect to the normal starting position)---this
code is followed by an integer $n$, and the break will occur with penalty
$10n$;
\item[\tt backup] denotes a backspace of one em;
\item[\tt cancel] obliterates any {\tt break\_space} or {\tt force} or
{\tt big\_force} 
tokens that immediately precede or follow it and also cancels any
{\tt backup} tokens that follow it;
\item[\tt indent] causes future lines to be indented one more em;
\item[\tt outdent] causes future lines to be indented one less em.
\item[\tt math\_rel] translates to \verb+\mathrel{+
\item[\tt math\_bin]translates to \verb+\mathbin{+
\item[\tt math\_op] translates to \verb+\mathop{+
\end{description}
The {\em only} key words that will work properly in math mode are {\tt
indent} and {\tt outdent}, so when you're defining the translations of
tokens you must use {\tt mathness~no} if your translations contain
other key words.
You may use any recognized key words in the translations of a
production; there the mathness is automatically taken care of for you.
\end{itemize}

Here are some example translations:
\begin{verbatim}
<"\\"-space>
<indent-force>
<"{\\let\\\\=\\bf"-space>
<"}"-indent-"{}"-space>
\end{verbatim}

\paragraph{Restricted translations}
In some cases, notably for a {\tt tangleto} description, translations
are {\em restricted}.
A restricted translation is never converted to typesetting code,
but is always converted to an ASCII string, usually for output by {\tt
TANGLE}, but sometimes for other things.
A restricted translation may contain only {\em quoted strings} and the
keywords {\tt space} and {\tt dash}.


\subsection{{\tt token} commands}
The syntax of the {\tt token} command is:
\begin{quote}
\tt  
\syntax{command} \produces~token \syntax{token-designator}
\syntax{token-descriptions}
\end{quote}
Where \syntax{token-descriptions} is a (possibly empty) list of token
descriptions.

\paragraph{Token descriptions}
The token descriptions are
\begin{itemize}\parindent=0pt
\item
{\tt tangleto \syntax{restricted translation}}

The \syntax{restricted translation} tells {\tt TANGLE} what program
text to write out for this token.
The only kinds of translation pieces valid in a restricted translation
are quoted strings and the special words {\tt space} and {\tt dash}.
If no {\tt tangleto} description is present, {\tt TANGLE} just writes
out the sequence of characters that constitute the token.

\item
{\tt translation \syntax{translation}}

Tells {\tt WEAVE} what translation to assign when making this token into
a scrap.
The self
marker~({\tt*}) stands for the sequence of characters that were read in to
make up the token.
The translation often defaults to \verb+translation <*>+; {Spider}
is set up to have this default initially.

\item
{\tt category \syntax{category-name}}

Tells {\tt WEAVE}  what category to assign when making this token into
a scrap.
If you're writing a {Spider} file, you may choose any category
names you like, subject only to the restriction that they not conflict
with other names  known to {Spider} (e.g.~predefined key words,
names of ilks, and so on).
Using category names that are identical to reserved words of the
target programming language (or reserved words of~C) is not only
supported, it is strongly encouraged, for clarity.
Also, when we get to the sample grammars later on, you will see some
other conventions we use for category names.

\item
{\tt mathness \syntax{mathness-indicator}}

where  \syntax{mathness-indicator} is {\tt yes}, {\tt no}, or {\tt
maybe}.
This indicates to {\tt WEAVE} whether the translation for this token
needs to be typeset in {\TeX}'s math mode or not, or whether it
doesn't matter.
When firing productions,
{\tt WEAVE} will place math shift characters~(\verb+$+) in the {\TeX}
text that guarantee the placement of tokens in the correct modes.
Tokens with the {\em empty translation} (\verb+<>+) should always have
{\tt mathness maybe}, lest they cause {\tt WEAVE} to place two
consecutive math shift characters.

\item 
{\tt name \syntax{token-name}}

This should only be necessary in debugging {Spider} or {\WEB}.
It causes the specified name to be attached to the token, so that a
programmer can search for that name in the C~code generated by 
{Spider}.

\end{itemize}

\paragraph{Token designators}
{Spider} recognizes the following token designators:
\begin{description}
\item[{\tt identifier}]
A {\tt token} command using this designator tells {\tt WEAVE} and {\tt
TANGLE} what to do with identifier tokens.
Unfortunately it is not possible to specify with {Spider} just
what an identifier is; that definition is hard-wired into {\tt WEAVE}
and {\tt TANGLE}.
An identifier is the longest string matching this regular expression%
\footnote{The reader unfamiliar with the Unix notation for regular
expressions should consult the {\it ed(1)} man page.}:
\begin{verbatim}
[a-zA-Z_][a-zA-Z0-9_]*
\end{verbatim}

\item[{\tt number}]
In the current implementation of {Spider} and {\tt WEAVE}, a {\tt token}
command using this designator covers the treatment of both numeric
constants and string constants.
Like the identifiers, the definitions of what constitutes a numeric or
string constant cannot be changed.
{\samepage
A numeric constant is the longest string matching%
\footnote{There  ought to be some kind of {\WEB} control sequence to
support floating point notation for those languages that have it.}:
\begin{verbatim}
[0-9]+(\.[0-9]*)?
\end{verbatim}
}
A string constant is the longest string matching
\begin{verbatim}
\"([^"]*\\\")*[^"]*\"|'[^@\]'|'\\.'|'@@'
\end{verbatim}
Carriage returns may appear in string constants if escaped by a
backslash~(\verb+\+).

\item[{\tt newline}]
A {\tt token} command using this descriptor tells {\tt WEAVE} and {\tt
TANGLE} how to treat a newline.
We'll see later how to make {\tt WEAVE} ignore newlines.

\item[{\tt pseudo\_semi}]
A {\tt token} command using this descriptor tells {\tt WEAVE} what to
do with the {\WEB} control sequence {\tt @;}.
This control sequence is always ignored by {\tt TANGLE}.

\item[\syntax{characters}]
where none of the characters is alphanumeric.
A {\tt token} command using this descriptor defines the sequence of
characters as a token, and tells {\tt WEAVE} and {\tt TANGLE} what to
do with that token.
A token may be a prefix of another token; {\tt WEAVE} and {\tt TANGLE}
will prefer the longer token to the shorter.
Thus, in a C~{\WEB}, \verb+==+ will be read as a single \verb+==+
token, not as two \verb+=+ tokens.
\end{description}




\subsection{Reserved word tokens}
Reserved words are attached to a particular {\em ilk} using the {\tt
reserved} command.
\begin{quote}
\tt  reserved \syntax{reserved-word} $[$ilk \syntax{ilk-name}$]$
\end{quote}
If you're writing a {Spider} file, you may choose any ilk
names you like, subject only to the restriction that they not conflict
with other names known to {Spider} (e.g.~predefined key words,
names of categories, and so on).
The convention, however, is to use ilk {\tt with\_like} for a reserved
word {\tt with}, and so on.%
\footnote{%
The existence of this convention seduced me into adding a pernicious
feature to {Spider}---if you omit the ilk from a {\tt reserved}
command, {Spider} will make an ilk name by appending {\tt \_like}
to the name of the reserved word.
Furthermore, if that ilk doesn't already exist, {Spider} will
construct one.
Don't use this feature.
}

The {\tt  ilk} and {\tt  token} commands have nearly identical syntax.
The syntax of the {\tt ilk} command is:
\begin{quote}\tt
\syntax{command} \produces~ilk \syntax{ilk-name}
\syntax{token-descriptions}
\end{quote}
In translations that appear in {\tt ilk} commands, the self
marker~({\tt *}) designates the string of characters making up the
reserved word, surrounded by \verb+\&{...}+, which makes the reserved
words appear in bold face.

\section{Syntax of the prettyprinting grammar}
Defining the tokens of a language is somewhat tedious, but it is
essentially straightforward, and the definition usually does not need
fine tuning.
When developing a new {\WEB} with {Spider}, you will spend most of
your time writing the grammar that tells {\tt WEAVE} how to reduce
scraps.
The grammar is defined as a sequence of context-sensitive productions. 
Each production has the form:
\begin{quote}
\tt 
\syntax{left context} [ \syntax{firing instructions} ] \syntax{right context}
\\\null\qquad
--> \syntax{left context} \syntax{target category} \syntax{right context}
\end{quote}
where the left and right contexts are (possibly empty) sequences of
scrap designators, the firing instructions are a sequence of scrap
designators and translations (containing at least one scrap
designator), and the target category is a category designator.
If the left and right contexts are both empty, the square brackets
({\tt []}) can be omitted, and the production is context free.
The left and right contexts must be the same on both sides of the {\tt
-->}.

What does the production mean?
Well, {\tt WEAVE} is trying to reduce a sequence of scraps.
So what {\tt WEAVE} does is look at the sequence, to find out whether
the left hand side of some production matches an initial subsequence
of the scraps.
{\tt WEAVE} picks the first matching production, and {\em fires} it,
reducing the scraps described in the firing instructions to a single
scrap, and it gives the new scrap the {\em target category}.
The translation of the new scrap is formed by concatenating the
translations in the {\em firing instructions}, where a scrap
designator stands for the translation of the designated scrap.

Here is the syntax that describes contexts, firing instructions, scrap
designators, and so on.
\begin{quote}
\tt
\syntax{left context} \produces~\syntax{scrap designators}\\
\syntax{right context} \produces~\syntax{scrap designators}\\
\syntax{firing instruction} \produces \syntax{scrap designator}\\
\syntax{firing instruction} \produces \syntax{translation}\\
\syntax{scrap designator} \produces~?\\
\syntax{scrap designator} \produces~\opt{!}\syntax{category name}\opt{*}\\
\syntax{scrap designator} 
   \produces~\opt{!}\syntax{category alternatives}\opt{*}\\
\syntax{category alternatives} 
   \produces~\rlap{(\syntax{optional alternatives}\syntax{category name})}\\
\syntax{optional alternative} \produces~\syntax{category name}|\\
\syntax{target category} \produces~\#\syntax{integer}\\
\syntax{target category} \produces~\syntax{category name}\\
\end{quote}

\paragraph{Matching the left hand side of a production}
When does a sequence of scraps match the left hand side of a
production?
For matching purposes, we can ignore the translations and the square
brackets~({\tt []}), and look at the left hand side just as a sequence
of scrap designators.
A sequence of scraps matches a sequence of scrap designators if and
only if each scrap on the sequence matches the corresponding scrap
designator.
Here are the rules for matching scrap designators (we can
ignore starring%
\footnote{A category name is said to be {\em starred} if it has the
optional {\tt *}.}%
):
\begin{itemize}
\item
Every scrap matches the designator {\tt ?}.
\item
A scrap matches \syntax{marked category} if and only if its category
is the same as the category of the designator.
\item
A scrap matches {\tt!}\syntax{marked category} if and only if its category
is {\em not} the same as the category of the designator.
(The {\tt !} indicates negation.)
\item
A scrap matches a list of category alternatives if and only if its
category is on the list of alternatives.
\item
A scrap matches a {\em negated} list of category alternatives if and
only if its category is {\em not} on the list of alternatives.
\end{itemize}

\paragraph{Firing a production}
Once a match is found, {\tt WEAVE} fires the production by replacing
the subsequence of scraps matching the firing instructions.
{\tt WEAVE} replaces this subsequence with a new scrap whose category
is the target category, and whose translation is the concatenation of
all the translations in the firing instructions.
(When the new translation is constructed, the
translations of the old scraps are included at the positions of the
corresponding scrap designators.)
If the target category is not given by name, but rather by
number~({\tt \#$n$}), {\tt WEAVE} will take the category of the $n$th
scrap in the subsequence that matches the left hand side of the
production, and make that the target category.

\subparagraph{Side effects of firing a production}
When a production fires, {\tt WEAVE} will {\em underline the
index entry} for the first identifier in any {\em starred} scrap.

\paragraph{If no initial subsequence matches any production}
If the initial subsequence of scraps does not match the left hand side
of any production, {\tt WEAVE} will try to match the subsequence
beginning with the second scrap, and so on, until a match is found.
Once a match is found, {\tt WEAVE} fires the production, changing its
sequence of scraps.
It then starts all over again at the beginning of the new sequence,
looking for a match.%
\footnote{
The implementation is better than that; {Spider} figures out just
how much {\tt WEAVE} must backtrack to get the same effect as
returning to the beginning.}
If {\em no} subsequence of the scraps matches any production, then the
sequence of scraps is irreducible, and {\tt WEAVE} writes out the
translations of the scraps, one at a time.

\section{Examples of {\tt WEAVE} grammars}
This all must seem very intimidating, but it's not really.
In this section we present some grammar fragments and explain what's
going on.

\paragraph{Short examples}
\begin{verbatim} 
? ignore_scrap --> #1
\end{verbatim} 
This production should appear in  every grammar, because Spidery {\tt
WEAVE} expects category \verb+ignore_scrap+ to exist with roughly this
semantics. 
(For example, all comments generate scraps of category {\tt
ignore\_scrap}.) 
Any scrap of category \verb+ignore_scrap+ essentially doesn't affect
the reduction of scraps: it is absorbed into the scrap to its left.

\begin{verbatim}
token newline category newline translation <>
newline --> ignore_scrap
\end{verbatim}
This token definition and production, combined with the previous
production, causes {\tt WEAVE} to ignore all newlines.

For this next example, from the C~grammar, you will need to know that 
{\tt math} represents a mathematical expression, {\tt semi} a
semicolon, and {\tt stmt} a statement or sequence of statements.
\begin{verbatim}
math semi --> stmt
stmt <force> stmt --> stmt
\end{verbatim}
The first production says that a mathematical expression, followed by
a semicolon, should be treated as a statement.
The second says that two statements can be combined to make a single
statement by putting a line break between them.

\paragraph{Expressions}
This more extended example shows the treatment of expressions in Awk.
This is identical to the treatment of expressions in C and in several
other languages.
We will use the following categories:
\begin{description}
\item[math] A mathematical expression
\item[binop] A binary infix operator
\item[unop] A unary prefix or postfix operator
\item[unorbinop] An operator that could be binary infix or unary
prefix
\end{description}
To show you how these might be used, here are some sample token
definitions using these categories:
\begin{verbatim}
token + category unorbinop
token - category unorbinop
token * category binop
token / category binop
token < category binop
token > category binop
token , category binop translation <",\\,"-opt-3>
token = category binop translation <"\\K">
token != translation <"\\I"> category binop
token == name eq_eq translation <"\\S"> category binop
token ++ name gt_gt category unop translation <"\\uparrow">
token -- name lt_lt category unop translation <"\\downarrow">
\end{verbatim}
Notice that the translation for the comma specifies a thin space and
an optional line break after the comma.
The translations of {\tt =},  {\tt !=}, and {\tt ==} 
produce~$\leftarrow$, $\ne$, and~$\equiv$. 

Here is the grammar for expressions.
\begin{verbatim}
math (binop|unorbinop) math --> math
(unop|unorbinop) math --> math
math unop --> math
math <"\\"-space> math --> math
\end{verbatim}
In Awk there is no concatenation operator; concatenation is by
juxtaposition.
The last production tells {\tt WEAVE} to insert a space between two
juxtaposed expressions.

So far we haven't dealt with parentheses, but that's easily done:
\begin{verbatim}
token ( category open
token ) category close
token [ category open
token ] category close
open math close --> math
\end{verbatim}


Now this grammar just given doesn't handle the Awk or C {\tt +=}
feature very well; {\tt x+=1} comes out as~$x+\leftarrow 1$, and {\tt
x/=2} is irreducible!
Here's the cure; first, we make a new category for assignment:
\begin{verbatim}
token = category equals translation <"\\K">
\end{verbatim}
And then we write productions that reduces assignment (possibly
preceded by another operator) to a binary operator:
\begin{verbatim}
<"\\buildrel"> (binop|unorbinop) <"\\over{"> equals <"}"> --> binop
equals --> binop
\end{verbatim}
Notice that, given the rules stated above, the second production can
fire only if {\tt equals} is {\em not} preceded by an operator.
On input~{\tt x+=1}, the first production fires, and we have the
translation~$x\buildrel+\over{\leftarrow} 1$. 

\paragraph{Conditional statements}
Here is the grammar for (possibly nested) conditional statements in
Awk.
\begin{verbatim}
if <"\\"-space> math --> ifmath
ifmath lbrace --> ifbrace
ifmath newline --> ifline
ifbrace <force> stmt --> ifbrace
ifbrace <outdent-force> close else <"\\"-space> if --> if
ifbrace <outdent-force> close else lbrace --> ifbrace
ifbrace <outdent-force> close else newline --> ifline
ifbrace <outdent-force> close --> stmt
(ifline|ifmath) <indent-force> stmt <outdent> --> stmt
\end{verbatim}
It relies on the following token definitions:
\begin{verbatim}
ilk if_like category if
reserved if
ilk else_like category else
reserved else
token { translation <"\\;\\{"-indent> category lbrace
token } translation <"\\}\\"-space> category close
token newline category newline translation <>
\end{verbatim}

\paragraph{Handling preprocessor directives in C}
Here is a simplified version of 
the grammar that handles C preprocessor directives.
It puts the directives on the left hand margin, and correctly handles
newlines escaped with backslashes.
(The full version is also able to distinguish {\tt <...>}
bracketing a file name from the use of the same symbols to mean ``less
than'' and ``greater than.'')
{\small\advance\hsize 1in
\begin{verbatim}
# control sequence \8 puts things on the left margin
<"\\8"> sharp <"{\\let\\\\=\\bf"-space> math <"}"-indent-"{}"-space> --> preproc
preproc backslash <force-"\\8\\hskip1em"-space> newline --> preproc
<force> preproc <force-outdent> newline --> ignore_scrap
preproc math --> preproc
newline --> ignore_scrap
\end{verbatim}
}
The \verb+\let+ in the first production makes the identifier following
the {\tt \#} come out in bold face.


\subsection{Using context-dependent productions}
So far we've been able to do a lot without using the
context-dependent features of {Spider} productions.
(For example, the entire {\tt spider} file for Awk is written using
only context-free productions.)
Now we'll show some examples that use the context-dependence.

In the grammar for Ada, a semicolon is used as a terminator for
statements.
But semicolons are also used as {\em separators} in parameter
declarations.
The first two productions here find the statements, but the third
production supersedes them when a semicolon is seen in a parenthesized
list.
\begin{verbatim}
semi --> terminator
math terminator --> stmt
open [ math semi ] --> open math
\end{verbatim}


\paragraph{Underlining the index entry for the name of a declared
function}
In SSL, function declarations begin with the type of the function
being declared, followed by the name of that function.
The following production causes the index entry for that function to
be underlined, so that we can look up the function name in the index
and easily find the section in which the function is declared:
\begin{verbatim} 
decl simp [ simp* ] --> decl simp math
\end{verbatim}
Where we've relied on
\begin{verbatim}
token identifier category simp mathness yes
\end{verbatim}


\paragraph{Conditional expressions}
Suppose we want to format conditional expressions (for example in C)
like this:
\begin{quote}
\syntax{condition}\\
\mbox{\qquad}$?$ \syntax{expression}\\
\mbox{\qquad}$:$ \syntax{expression}
\end{quote}
The problem is that it's hard to know when the conditional expression
ends.
It's essentially a question of precedence, and what we're going to do
is look ahead until we see an operator with sufficiently low
precedence that it terminates a conditional expression.
In SSL a conditional expression can be terminated by a semicolon, a
right parenthesis, a comma, or a colon.
We'll use the {\em right context} to do the lookahead.
{\small
\begin{verbatim}
token ? translation <"\\?"> category question
token : category colon

<indent-force> question math <force> colon --> condbegin
[ condbegin math <outdent> ] (semi|close|comma|colon) --> math (semi|close|comma|colon)
\end{verbatim}
}

\subsection{Debugging a prettyprinting grammar}
{\tt WEAVE} has two tracing modes that can help you debug a
prettyprinting grammar.
The control sequence {\tt @1} turns on partial tracing, and {\tt @2}
turns on a full trace.  
{\tt @0} turns tracing back off again.
In the partial tracing mode, {\tt WEAVE} applies all the productions
as many times as possible, and then it prints out the irreducible
scraps that remain.
If the scraps reduce to a single scrap,  no diagnostics are printed.


When a scrap is printed, {\tt WEAVE} prints a leading
{\tt+}~or~{\tt-}, the name of the category of that scrap, and a
trailing {\tt+}~or~{\tt-}.
The {\tt+} indicates that {\TeX} should be in math mode, and the
{\tt-} that {\TeX} should not be in math mode, at the beginning and
end of the scrap's translation, respectively.
(You can see the translations by looking at the {\tt.tex} file, since
that's where they're written out.)

For beginners, the full trace is more helpful.
It prints out the following information every time a production is
fired:
\begin{itemize}
\item
The number of the production just fired (from {\tt productions.list});
\item
The sequence of scraps {\tt WEAVE} is now trying to reduce;
\item
A {\tt*} indicating what subsequence {\tt WEAVE} will try to reduce
next.
\end{itemize}
A good way to understand how prettyprinting grammars work is to take
a {\tt productions.list} file, and look at a full trace of the
corresponding {\tt WEAVE}.
Or, if you prefer, you can simulate by hand the action of {\tt WEAVE}
on a sequence of scraps.

\section{The rest of the {Spider} language}
The tokens and the grammar are not quite the whole story.
Here's  the rest of the truth about what you can do with {Spider}.

\subsection{Naming the target language}
When a Spidery {\tt WEAVE} or {\tt TANGLE} starts up, it prints the
target language for which it was generated, and the date and time of
the generation.
The {\tt language} command is used to identify the language being
targeted.
Its syntax is
\begin{quote}
\tt language \syntax{language-name}
\opt{extension \syntax{extension-name}}\\
\mbox{\qquad\qquad}\opt{version \syntax{version-name}}
\end{quote}
The extension name is the extension used (in place of {\tt .web}) by
{\tt TANGLE} to write out the program text for the unnamed module.
The extension is also used to construct a language-specific file of
{\TeX} macros to be used by {\tt WEAVE}, so different languages should
always have different extensions.
If the extension is  not given it defaults to the language name.
If the version information is given, it too will be printed out at
startup.

The {\tt c.spider} file I use for Unix has
\begin{verbatim}
language C extension c
\end{verbatim}


\subsection{Defining {\TeX} macros}
In addition to the ``kernel'' {\WEB} macros stored in {\tt
webkernel.tex}, you may want to create some {\TeX} macros of your
own for use in translations.
Any macro definitions you put between lines saying {\tt macros begin}
and {\tt macros end} will be included verbatim in the {\TeX} macro
file for this language.
That macro  file will automatically be \verb+\input+ by every {\TeX}
file generated by this {\tt WEAVE}.

For example, the C grammar includes productions to handle preprocessor
directives.
These  directives may include file names that are delimited by angle
brackets.
I wanted to use the abbreviations \verb+\LN+ and \verb+\RN+ for left
and right angle brackets, so I included
\begin{verbatim}
macros begin
\let\LN\langle
\let\RN\rangle
macros end
\end{verbatim}
in the {\tt c.spider} file.

\subsection{Setting default token information}
It's possible to set default values for the {\tt translation} and {\tt
mathness} properties of tokens, so that they don't have to be
repeated.
This is done with the {\tt default} command, whose syntax is:
\begin{quote}
\tt
default \syntax{token descriptions}
\end{quote}
The initial defaults (when {Spider} begins execution) are {\tt
translation~<*>} and {\tt mathness~maybe}.

\subsection{Specifying the treatment of modules}
{\WEB} introduces a new kind of token that isn't in any programming
language, and that's the module name ({\tt @<...@>} or {\tt @(...@>}).
{\tt TANGLE}'s job is to convert the module names to program text, and
when {\tt TANGLE} is finished no module names remain.
But {\tt WEAVE} has to typeset the module names, and we need to tell
{\tt WEAVE} what category to give a scrap created from a module name.
We allow two different categories, one for the definition of the
module name (at the beginning of a module), and one for a use of a
module name.
{\samepage
The syntax of the {\tt module} command is:
\begin{quote}
\tt
module \opt{definition \syntax{category name}} 
\opt{use \syntax{category name}}
\end{quote}
}

The {\tt c.spider} file contains the line
\begin{verbatim}
module definition decl use math
\end{verbatim}

\subsection{Determining the at sign}
When generating a {\WEB} system with {Spider}, you're not required
to use ``{\tt @}'' as the ``magic at sign'' that introduces {\WEB}
control sequences.
By convention, however, we use ``{\tt @}'' unless that is deemed
unsuitable.
If ``{\tt @}'' is unsuitable, we use ``{\tt \#}.''
Since {Spider} writes C~{\WEB} code for {\tt WEAVE} and {\tt
TANGLE}, it writes a lot of {\tt @} signs.
I didn't when to have to escape each one, so I chose
``{\tt \#}'' for Awk~{\WEB}'s at sign:
\begin{verbatim}
at_sign #
\end{verbatim}
The at sign defaults to ``{\tt @}'' if left unspecified.

\paragraph{Changing control sequences}
Changing the at sign changes the meaning of one or two control
sequences.
This is more easily illustrated by example than explained.
Suppose we change the at sign to {\tt\#}.
Then in the resulting {\WEB} two control sequences have new meanings:
\begin{description}
\item[{\tt \#\#}]
Stands for a {\tt \#} in the input, by analogy with {\tt @@} in normal
{\WEB}.
You will need this when defining {\TeX} macros that take parameters.
\item[{\tt \#@}]
This is the new name of the control sequence normally represented by
{\tt@\#}.
 You would use {\tt\#@} to get a line break followed by vertical
white space.
\end{description}
If you change the at sign to something other than {\tt@}~or~{\tt\#},
the above will still hold provided you substitute your at sign for
{\tt\#}.


\subsection{Comments in the programming language}
We have to tell {\tt WEAVE} and {\tt TANGLE} how to recognize
comments in our target programming language, since comment text is
treated as {\TeX} text by {\tt WEAVE} and is ignored by {\tt TANGLE}.
The syntax of the {\tt comment} command is
\begin{quote}
\tt
comment begin \syntax{restricted translation} \\
\null\qquad end $(\syntax{restricted translation}|{\tt newline})$
\end{quote}
The restricted translations can include  only  quoted
strings, {\tt space}, and 
{\tt dash}.
They give the character sequences that begin and end comments.
If comments end with newlines the correct incantation is {\tt end
newline}.

If the comment character is the same as the at sign, it has to be
doubled in the {\WEB} file to have any effect.
For reasons that I've forgotten, {Spider} is too dumb
to figure this out and {\em you must double the comment character in
the {Spider} file}.
This is not totally unreasonable since any at sign that actually appears in a
{\WEB}  file will have to be double to be interpreted correctly.

{\tt WEAVE} uses the macros \verb+\commentbegin+ and
\verb+\commentend+ at the beginning and end of comments, so you can
define these to be whatever you like (using the {\tt macros} command)
if you don't like {Spider}'s defaults.
{Spider} is smart enough to escape {\TeX}'s special characters in
coming up with these defaults.

Here's a real-world ugly example of how things really are, from the
{\tt spider} file for Awk:
\begin{verbatim}
comment begin <"##"> end newline
macros begin
\def\commentbegin{\#} % we don't want \#\#
macros end
\end{verbatim}

\subsection{Controlling line numbering}
 A compiler doesn't get to see  {\WEB} files directly; it has to
read the output of {\tt TANGLE}.
Error messages labelled with line numbers from a tangled file aren't
very helpful, so  Spidery {\tt TANGLE} does something to improve the
situation: it writes {\tt \#line} directives into its output, in the
manner of the C~preprocessor.
({\tt TANGLE} also preserves the line breaking of the {\WEB} source,
so that the {\tt \#line} information will be useful.)
For systems like Unix with {\tt cc} and {\tt dbx}, both compile-time
and run-time debugging can be done in terms of the {\WEB} source, and
the intermediate programming language source need never be consulted.

Not all compilers support line numbering with {\tt \#line} directives,
so {Spider} provides a {\tt line} command to change the format of
the {\tt \#line} directives.
If your compiler doesn't support {\tt \#line}, you can use the {\tt
line} command to turn the line number information into a comment.%
\footnote{%
There should be a command that turns off line numbering.%
}
The syntax is:
\begin{quote}
\tt
line begin \syntax{restricted translation} end \syntax{restricted translation}
\end{quote}
The {\tt begin} translation tells what string to put in front of the file
name and line number information; the {\tt end} translation tells what to
put afterward.
The defaults (which are set for C) are
\begin{verbatim}
line begin <"#line"> end <"">
\end{verbatim}
Here's an example from the Ada~{Spider} file, which makes the line
number information into an Ada comment:
\begin{verbatim}
line begin <"--"-space-"line"> end <"">
\end{verbatim}




\subsection{Showing the date of generation}
When Spidery {\tt WEAVE} and {\tt TANGLE} start up, they print the
date and time at which their {Spider} file was processed.
This is done through the good offices of {Spider}'s {\tt date}
command, which is
\begin{quote}
\tt
date \syntax{date}
\end{quote}
where \syntax{date} looks like {\tt "Fri Dec 11 11:31:18 EST 1987"} or
some such.
Normally you never need to use the {\tt date} command, because one is
inserted automatically by the {Spider} tools, but if you're
porting {Spider} to a non-Unix machine you need to know about it.




\section{Spider's error messages}
{Spider} makes a lot of attempts to detect errors in a 
{Spider} specification.
{Spider}'s error messages are intended to be self-explanatory, but
I don't know how well they succeed.
In case you run into trouble, here are the error
conditions {Spider} tries to detect:
\begin{itemize}
\item 
 Garbled commands, lines with bad fields in them, or commands with
unused fields.
Any command with a field {Spider} can't follow or with an extra
field is ignored from the bad field onward, but the earlier fields may
have an effect.
Any production with a bad field or other error is dropped completely.
\item
Missing {\tt language} command.
\item
{\tt macros} or {\tt comment} command before {\tt language} command.
{Spider}  uses the {\tt extension} information
from the  {\tt language} command to determine the name of the file to
which the macros will be written, and  the {\tt comment} command
causes {Spider} to write macros telling {\TeX} what to do at the
beginning and end of comments.
\item
Contexts don't match on the left and right sides of a production.
\item
A numbered target token doesn't fall in the range defined by the
left hand side of its production.
\item
Some category is never {\em appended}.
This means there is no  way to create a scrap with this category.
{Spider} only looks to see that each
category appears at least once as the category of some token or as the
category of the target token in some production, so {Spider} might
fail to detect this condition (if there is some production that can
never fire).
\item
Some category is never {\em reduced}.
This  means that the category never appears in a scrap
designator from the firing instructions of a production.
If a category is never reduced, {Spider} only issues a warning,
and does not halt the compilation process with an error.

The append
and reduce checks will usually catch you if you misspell a category
name.
\item
You defined more tokens than {\tt WEAVE} and {\tt TANGLE} can handle.
\item
You forgot token information for identifiers, numeric
constants, newlines, pseudo-semicolons~({\tt @;}), module definitions,
or module uses.
\item
Some ilk has no translation, or there is some ilk of which there are
no reserved words.
\end{itemize}




\section{{Spider}'s output files}
{Spider} writes many output files, and you may want to look at
them to figure out what's going on.
Here is a partial list (you can find a complete list by consulting
{\tt spider.web}):
\begin{description}
\item[\tt cycle.test]
Used to try to detect potential loops in the grammar.
Such loops can cause {\tt WEAVE} to run indefinitely (until it runs
out of memory) on certain inputs.
Discussed below with the {Spider} tools.
\item[\tt spider.slog]
A verbose discussion of everything {Spider} did while it was
processing your file.
To be consulted when things go very wrong.
\item[\tt *web.tex]
The macros specific to the generated {\WEB}.
\item[\tt productions.list]
A numbered list of all the productions.
This list is invaluable when you are trying to debug a grammar using
Spidery {\tt WEAVE}'s tracing facilities ({\tt @2}).
\end{description}




\section{Using {Spider} to make {\WEB} (the {Spider} tools)}
Many of the {Spider} tools do error checking, like:
\begin{itemize}
\item
Check to see there are no duplicate names among the categories, ilks,
and translation keywords.
\item
Check the translation keywords against a list of those recognized by
{\tt WEAVE}, and yelps if trouble happens.
\item
Try to determine whether there is a ``production cycle'' that could
cause {\tt WEAVE} to loop infinitely by firing the productions in the
cycle one after another.
\end{itemize}

I'm not going to say much about how to do all this, or how to make
{\tt WEAVE} and {\tt TANGLE}; instead I'm going to show you a {\tt
Makefile} and comment on it a little bit.
Since right now Spidery {\tt WEB} is available only on Unix systems,
chances are you have the {\tt Makefile} and can just type ``{\tt
make~tangle}'' or ``{\tt make~weave}.'' 
If not, reading the Makefile is still your best bet to figure out what
the tools do.

We assume that you are making {\tt WEAVE} and {\tt TANGLE} in some
directory, and that the ``master sources'' for Spidery {\WEB} are kept
in some other directory.
Some of the  {\tt Makefile} macros deserve special mention:
\begin{description}
\renewcommand{\makelabel}[1]{{\tt#1}\hfil}
\item[THETANGLE]
Name of the {\tt TANGLE} we will generate.
\item[THEWEAVE]
Name of the {\tt WEAVE} we will generate.
\item[SPIDER]
Name of the {Spider} input file.
\item[DEST]
The directory in which the executables defined by \verb+$(TANGLE)+ and
\verb+$(WEAVE)+ will be placed.
\item[WEBROOT]
The directory that is the root of the Spidery {\WEB} distribution.
\item[MASTER]
The location of the ``master sources.''
This should always be different from the directory in which {\tt make}
is called, or havoc will result.
\item[CTANGLE]
The name of the program used to tangle C code.
\item[AWKTANGLE]
The name of the program used to tangle Awk code.
\item[MACROS]
The name of a directory in which to put {\TeX} macro definitions (a
{\tt *web.tex} file.
\end{description}

Usually we will only be interested in two commands: ``\/{\tt
make~weave}'' and ``\/{\tt make~tangle}.''
It's safe to use ``\/{\tt make~clean}'' only if you use the current
directory for nothing exception spidering; ``\/{\tt make~clean}'' is
really vicious.

The line that's really of interest is the line showing the dependency
for {\tt grammar.web}.
First we run {Spider}.
Then we check for bad translation keywords and for potential cycles in
the prettyprinting grammar.
We check for duplicate names, and then finally (if everything else
works), we put the {\tt *web.tex} file in the right place.

Here's \verb+$(MASTER)/WebMakefile+:
\begingroup\small
\begin{verbatim}
# Copyright 1989 by Norman Ramsey and Odyssey Research Associates.
# To be used for research purposes only.
# For more information, see file COPYRIGHT in the parent directory.

HOME=/u/nr#			# Make no longer inherits environment vars
THETANGLE=tangle
THEWEAVE=weave
SPIDER=any.spider
#
DVI=dvi
CFLAGS=-DDEBUG -g -DSTAT

# CPUTYPE is a grim hack that attempts to solve the problem of multiple
# cpus sharing a file system.  In my environment I have to have different
# copies of object and executable for vax, sun3, next, iris, and other 
# cpu types.  If you will be using Spidery WEB in a homogeneous processor
# environment, you can just set CPUTYPE to a constant, or eliminate it 
# entirely.  
#
# In my environment, the 'cputype' program returns a string that
# describes the current processor.  That means that the easiest thing
# for you to do is to define a 'cputype' program that does something
# sensible.  A shell script that says 'echo "vax"' is fine.

CPUTYPE=`cputype`

# Change the following three directories to match your installation
#
# the odd placement of # is to prevent any trailing spaces from slipping in

WEBROOT=$(HOME)/web/src# 	# root of the WEB source distribution
DEST=$(HOME)/bin/$(CPUTYPE)#	 	# place where the executables go
MACROS=$(HOME)/tex/macros# 	# place where the macros go

MASTER=$(WEBROOT)/master# 	# master source directory
OBDIR=$(MASTER)/$(CPUTYPE)#	# common object files

TANGLESRC=tangle
CTANGLE=ceetangle -I$(MASTER)
CWEAVE=ceeweave -I$(MASTER)
AWKTANGLE=awktangle -I$(MASTER)
COMMONOBJS=$(OBDIR)/common.o $(OBDIR)/pathopen.o
COMMONC=$(MASTER)/common.c $(MASTER)/pathopen.c
COMMONSRC=$(COMMONC) $(MASTER)/spider.awk



# Our purpose is to make tangle and weave

web: tangle weave

tangle: $(COMMONOBJS) $(TANGLESRC).o
	cc $(CFLAGS) -o $(DEST)/$(THETANGLE) $(COMMONOBJS) $(TANGLESRC).o

weave: $(COMMONOBJS) weave.o
	cc $(CFLAGS) -o $(DEST)/$(THEWEAVE) $(COMMONOBJS) weave.o


source: $(TANGLESRC).c $(COMMONSRC) # make tangle.c and common src, then clean
	if [ -f WebMakefile ]; then exit 1; fi  # don't clean the master!
	if [ -f spiderman.tex ]; then exit 1; fi # don't clean the manual
	-rm -f tangle.web weave.* common.* # remove links that may be obsolete
	-rm -f *.unsorted *.list grammar.web outtoks.web scraps.web 
	-rm -f cycle.test spider.slog
	-rm -f *.o *.tex *.toc *.dvi *.log *.makelog *~ *.wlog *.printlog

# Here is how we make the common stuff

$(MASTER)/common.c: $(MASTER)/common.web # no change file
	$(CTANGLE) $(MASTER)/common 

$(OBDIR)/common.o: $(MASTER)/common.c
	cc $(CFLAGS) -c $(MASTER)/common.c
	mv common.o $(OBDIR)


$(MASTER)/pathopen.c: $(MASTER)/pathopen.web # no change file
	$(CTANGLE) $(MASTER)/pathopen 
	mv pathopen.h $(MASTER)

$(OBDIR)/pathopen.o: $(MASTER)/pathopen.c
	cc $(CFLAGS) -c $(MASTER)/pathopen.c
	mv pathopen.o $(OBDIR)


## Now we make the tangle and weave source locally

$(TANGLESRC).c: $(MASTER)/$(TANGLESRC).web $(MASTER)/common.h grammar.web
	-/bin/rm -f $(TANGLESRC).web
	ln $(MASTER)/$(TANGLESRC).web $(TANGLESRC).web
#	chmod -w $(TANGLESRC).web
	$(CTANGLE) $(TANGLESRC)

weave.c: $(MASTER)/weave.web $(MASTER)/common.h grammar.web 
	-/bin/rm -f weave.web
	ln $(MASTER)/weave.web weave.web
#	chmod -w weave.web
	$(CTANGLE) weave 

## Here's where we run SPIDER to create the source

grammar.web: $(MASTER)/cycle.awk $(MASTER)/spider.awk $(SPIDER)
	echo "date" `date` | cat - $(SPIDER) | awk -f $(MASTER)/spider.awk
	cat $(MASTER)/transcheck.list trans_keys.unsorted | awk -f $(MASTER)/transcheck.awk
	awk -f $(MASTER)/cycle.awk < cycle.test
	sort *.unsorted | awk -f $(MASTER)/nodups.awk
	mv *web.tex $(MACROS)

## We might have to make spider first.

$(MASTER)/spider.awk: $(MASTER)/spider.web
	$(AWKTANGLE) $(MASTER)/spider
	mv cycle.awk nodups.awk transcheck.awk $(MASTER)
	rm junk.list


# $(MASTER)/cycle.awk: $(MASTER)/cycle.web # making spider also makes cycle
# 	$(AWKTANGLE) $(MASTER)/cycle


# This cleanup applies to every language

clean:
	if [ -f WebMakefile ]; then exit 1; fi # don't clean the master!
	if [ -f spiderman.tex ]; then exit 1; fi # don't clean the manual
	-rm -f tangle.* weave.* common.* # remove links that may be obsolete
	-rm -f *.unsorted *.list grammar.web outtoks.web scraps.web 
	-rm -f cycle.test spider.slog
	-rm -f *.c *.o *.tex *.toc *.dvi *.log *.makelog *~ *.wlog *.printlog



# booting the new distribution
boot:
	cd ../master; rm -f *.o; for i in $(COMMONC); do \
		cc $(CFLAGS) -c $$i; \
		mv *.o $(OBDIR) ; \
	done; cd ../c
	cc $(CFLAGS) -c $(TANGLESRC).c; \
	cc $(CFLAGS) -o $(DEST)/$(THETANGLE) $(COMMONOBJS) $(TANGLESRC).o

 
\end{verbatim}
\endgroup

\section{Getting your own  Spidery {\tt WEB}}
At this time, Spidery {\tt WEB} has been tested only on Unix machines.
It should be easy to port to any machine having a C compiler and an
Awk interpreter, but undoubtedly some changes will be necessary.
The full {Spider} distribution, including this manual, is available by
anonymous {\tt ftp} from {\tt princeton.edu:~ftp/pub/spiderweb.tar.Z}.
You should pick a directory to install {Spider} in, untar the
distribution, and follow the directions in the README file.
The directory you have picked becomes {\tt WEBROOT}.

If the {\tt Makefile} in the distribution differs from the one given
above, the one in the distribution should be considered the correct
one. 




\section{A real {Spider} file}
I have tried to use real examples to illustrate the use of {Spider}.
I include here, as an extended example, the complete {Spider} file for
the Awk language. 
Those who cannot easily study the distribution may find it useful to
study this.
\begingroup\small
\begin{verbatim}
# Copyright 1989 by Norman Ramsey and Odyssey Research Associates.
# To be used for research purposes only.
# For more information, see file COPYRIGHT in the parent directory.

language AWK extension awk

at_sign #

module definition stmt use stmt
# use as stmt is unavoidable since tangle introduces line breaks

comment begin <"##"> end newline
macros begin
\def\commentbegin{\#} % we don't want \#\#
macros end

line begin <"#line"> end <"">

default translation <*> mathness yes

token identifier category math mathness yes
token number category math mathness yes
token newline category newline translation <> mathness maybe
token pseudo_semi category ignore_scrap mathness no translation <opt-0>

token \ category backslash translation <> mathness maybe
token + category unorbinop
token - category unorbinop
token * category binop
token / category binop
token < category binop
token > category binop
token >> category binop translation <"\\GG">
token = category equals translation <"\\K">
token ~ category binop translation <"\\TI">
token !~ category binop translation <"\\not\\TI">
token & category binop translation <"\\amp">
token % translation <"\\%"> category binop
token ( category open
token [ category lsquare
token ) category close
token ] category close
token { translation <"\\;\\{"-indent> category lbrace
token } translation <"\\}\\"-space> category close
token , category binop translation <",\\,"-opt-3>

token ; category semi translation <";"-space-opt-2> mathness no
# stuff with semi can be empty in for statements
open semi --> open
semi semi --> semi
semi close --> close
semi --> binop

# token : category colon
# token | category bar
token != name not_eq translation <"\\I"> category binop
token <= name lt_eq translation <"\\L"> category binop
token >= name gt_eq translation <"\\G"> category binop
token == name eq_eq translation <"\\S"> category binop
token && name and_and translation <"\\W"> category binop
token || name or_or translation <"\\V"> category binop
# token -> name minus_gt translation <"\\MG"> category binop
token ++ name gt_gt category unop translation <"\\uparrow">
token -- name lt_lt category unop translation <"\\downarrow">
# preunop is for unary operators that are prefix only
token $ category preunop translation <"\\DO"> mathness yes

default mathness yes translation <*>

ilk pattern_like category math
reserved BEGIN ilk pattern_like
reserved END ilk pattern_like

ilk if_like category if
reserved if
ilk else_like category else
reserved else

ilk print_like category math
# math forces space between this and other math...
reserved print ilk print_like
reserved printf ilk print_like
reserved sprintf ilk print_like


ilk functions category unop mathness yes
reserved length ilk functions
reserved substr ilk functions
reserved index ilk functions
reserved split ilk functions
reserved sqrt ilk functions
reserved log  ilk functions
reserved exp ilk functions
reserved int ilk functions

ilk variables category math mathness yes
reserved NR ilk variables
reserved NF ilk variables
reserved FS ilk variables
reserved RS ilk variables
reserved OFS ilk variables
reserved ORS ilk variables

ilk for_like category for
reserved for ilk for_like
reserved while ilk for_like

ilk in_like category binop translation <math_bin-*-"}"> mathness yes
# translation <"\\"-space-*-"\\"-space>
reserved in ilk in_like

ilk stmt_like category math
reserved break ilk stmt_like
reserved continue ilk stmt_like
reserved next ilk stmt_like
reserved exit ilk stmt_like


backslash newline --> math
# The following line must be changed to make a backslash
backslash <"\\backslash"> --> math

math (binop|unorbinop) math --> math
<"\\buildrel"> (binop|unorbinop) <"\\over{"> equals <"}"> --> binop
equals --> binop
(unop|preunop|unorbinop) math --> math
# unorbinop can only act like unary op as *prefix*, not postfix
math unop --> math
math <"\\"-space> math --> math
# concatenation

math newline --> stmt
newline --> ignore_scrap

stmt <force> stmt --> stmt

(open|lsquare) math close --> math

math lbrace --> lbrace
lbrace <force> stmt --> lbrace
lbrace <outdent-force> close --> stmt

if <"\\"-space> math --> ifmath
ifmath lbrace --> ifbrace
ifmath newline --> ifline
ifbrace <force> stmt --> ifbrace
ifbrace <outdent-force> close else <"\\"-space> if --> if
ifbrace <outdent-force> close else lbrace --> ifbrace
ifbrace <outdent-force> close else newline --> ifline
ifbrace <outdent-force> close --> stmt
(ifline|ifmath) <indent-force> stmt <outdent-force> else <"\\"-space> if --> if
(ifline|ifmath) <indent-force> stmt <outdent-force> else lbrace --> ifbrace
(ifline|ifmath) <indent-force> stmt <outdent-force> else newline --> ifline
(ifline|ifmath) <indent-force> stmt <outdent-force> else --> ifmath
(ifline|ifmath) <indent-force> stmt <outdent> --> stmt

for <"\\"-space> math --> formath
formath lbrace --> forbrace
formath newline --> forline
forbrace <force> stmt --> forbrace
forbrace <outdent-force> close --> stmt
(forline|formath) <indent-force> stmt <outdent> --> stmt


? ignore_scrap --> #1
\end{verbatim}
\endgroup



\section{Bibliography}
\begin{thebibliography}{Knuth~999}
\bibitem[Bentley~87]{bentley:pearls}
Jon L. Bentley, ``Programming Pearls,'' {\sl Communications of the
ACM}~{\bf 29:5}(May 1986), 364--?, and {\bf 29:6}(June 1986),
471--483.
Two columns on literate programming.
The first is an introduction, and the second is an extended example by
Donald Knuth, with commentary by Douglas MacIlroy.
\bibitem[Knuth~83]{knuth:web}
Donald~E. Knuth,
``The {{\tt WEB}} system of structured documentation''
 Technical Report 980, Stanford Computer Science, Stanford,
  California, September 1983.
The manual for the original {\tt WEB}.
\bibitem[Knuth~84]{knuth:literate-programming}
Donald E. Knuth, ``Literate Programming,'' {\sl The Computer Journal}
{\bf 27:2}(1984), 97--111.
The original introduction to literate programming with {\WEB}.
\bibitem[Levy~87]{levy:cweb}
Silvio Levy, ``Web Adapted to C, Another Approach'' {\sl TUGBoat}
{\bf 8:2}(1987), 12--13.
A short note about the C implementation of {\WEB}, from which Spidery
{\WEB} is descended.
\bibitem[Sewell~89]{sewell:weaving}
Wayne Sewell, ``Weaving a program: Literate programming in {\tt
WEB},''
Van Nostrand Reinhold, 1989.
\end{thebibliography}


\end{document}
